\documentclass[english,man]{apa6}

\usepackage{amssymb,amsmath}
\usepackage{ifxetex,ifluatex}
\usepackage{fixltx2e} % provides \textsubscript
\ifnum 0\ifxetex 1\fi\ifluatex 1\fi=0 % if pdftex
  \usepackage[T1]{fontenc}
  \usepackage[utf8]{inputenc}
\else % if luatex or xelatex
  \ifxetex
    \usepackage{mathspec}
    \usepackage{xltxtra,xunicode}
  \else
    \usepackage{fontspec}
  \fi
  \defaultfontfeatures{Mapping=tex-text,Scale=MatchLowercase}
  \newcommand{\euro}{€}
\fi
% use upquote if available, for straight quotes in verbatim environments
\IfFileExists{upquote.sty}{\usepackage{upquote}}{}
% use microtype if available
\IfFileExists{microtype.sty}{\usepackage{microtype}}{}

% Table formatting
\usepackage{longtable, booktabs}
\usepackage{lscape}
% \usepackage[counterclockwise]{rotating}   % Landscape page setup for large tables
\usepackage{multirow}		% Table styling
\usepackage{tabularx}		% Control Column width
\usepackage[flushleft]{threeparttable}	% Allows for three part tables with a specified notes section
\usepackage{threeparttablex}            % Lets threeparttable work with longtable

% Create new environments so endfloat can handle them
% \newenvironment{ltable}
%   {\begin{landscape}\begin{center}\begin{threeparttable}}
%   {\end{threeparttable}\end{center}\end{landscape}}

\newenvironment{lltable}
  {\begin{landscape}\begin{center}\begin{ThreePartTable}}
  {\end{ThreePartTable}\end{center}\end{landscape}}

  \usepackage{ifthen} % Only add declarations when endfloat package is loaded
  \ifthenelse{\equal{\string man}{\string man}}{%
   \DeclareDelayedFloatFlavor{ThreePartTable}{table} % Make endfloat play with longtable
   % \DeclareDelayedFloatFlavor{ltable}{table} % Make endfloat play with lscape
   \DeclareDelayedFloatFlavor{lltable}{table} % Make endfloat play with lscape & longtable
  }{}%



% The following enables adjusting longtable caption width to table width
% Solution found at http://golatex.de/longtable-mit-caption-so-breit-wie-die-tabelle-t15767.html
\makeatletter
\newcommand\LastLTentrywidth{1em}
\newlength\longtablewidth
\setlength{\longtablewidth}{1in}
\newcommand\getlongtablewidth{%
 \begingroup
  \ifcsname LT@\roman{LT@tables}\endcsname
  \global\longtablewidth=0pt
  \renewcommand\LT@entry[2]{\global\advance\longtablewidth by ##2\relax\gdef\LastLTentrywidth{##2}}%
  \@nameuse{LT@\roman{LT@tables}}%
  \fi
\endgroup}


\ifxetex
  \usepackage[setpagesize=false, % page size defined by xetex
              unicode=false, % unicode breaks when used with xetex
              xetex]{hyperref}
\else
  \usepackage[unicode=true]{hyperref}
\fi
\hypersetup{breaklinks=true,
            pdfauthor={},
            pdftitle={Modeling Memory: Exploring the Relationship Between Word Overlap and Single Word Norms when Predicting Relatedness Judgments and Retrieval},
            colorlinks=true,
            citecolor=blue,
            urlcolor=blue,
            linkcolor=black,
            pdfborder={0 0 0}}
\urlstyle{same}  % don't use monospace font for urls

\setlength{\parindent}{0pt}
%\setlength{\parskip}{0pt plus 0pt minus 0pt}

\setlength{\emergencystretch}{3em}  % prevent overfull lines

\ifxetex
  \usepackage{polyglossia}
  \setmainlanguage{}
\else
  \usepackage[english]{babel}
\fi

% Manuscript styling
\captionsetup{font=singlespacing,justification=justified}
\usepackage{csquotes}
\usepackage{upgreek}

 % Line numbering
  \usepackage{lineno}
  \linenumbers


\usepackage{tikz} % Variable definition to generate author note

% fix for \tightlist problem in pandoc 1.14
\providecommand{\tightlist}{%
  \setlength{\itemsep}{0pt}\setlength{\parskip}{0pt}}

% Essential manuscript parts
  \title{Modeling Memory: Exploring the Relationship Between Word Overlap and
Single Word Norms when Predicting Relatedness Judgments and Retrieval}

  \shorttitle{Judgments and Recall}


  \author{Nicholas P. Maxwell\textsuperscript{1}~\& Erin M. Buchanan\textsuperscript{1}}

  \def\affdep{{"", ""}}%
  \def\affcity{{"", ""}}%

  \affiliation{
    \vspace{0.5cm}
          \textsuperscript{1} Missouri State University  }

  \authornote{
    \newcounter{author}
    Nicholas P. Maxwell is a graduate student at Missouri State University.
    Erin M. Buchanan is an Associate Professor of Psychology at Missouri
    State University.

                      Correspondence concerning this article should be addressed to Nicholas P. Maxwell, 901 S. National Ave, Springfield, MO, 65897. E-mail: \href{mailto:maxwell270@live.missouristate.edu}{\nolinkurl{maxwell270@live.missouristate.edu}}
                          }


  \abstract{This study examined the interactive relationship between semantic,
thematic, and associative word pair strength in the prediction of item
relatedness judgments and cued-recall performance. Previously, we found
significant three-way interactions between associative, semantic,
thematic word overlap when predicting participant judgment strength and
recall performance ({\textbf{???}}), expanding upon previous work by
Maki (2007a). In this study, we first seek to replicate findings from
the original study using a novel stimuli set. Second, this study will
further explore the nature of the structure of memory, by investigating
the effects of single concept information (i.e., word frequency,
concreteness, etc.) on relatedness judgments and recall accuracy. We
hypothesize that associative, semantic, and thematic memory networks are
interactive in their relationship to judgments and recall, even after
controlling for base rates of single concept information, implying a set
of interdependent memory systems used for both cognitive processes.}
  \keywords{judgments, memory, association, semantics, thematics \\

    
  }





\usepackage{amsthm}
\newtheorem{theorem}{Theorem}
\newtheorem{lemma}{Lemma}
\theoremstyle{definition}
\newtheorem{definition}{Definition}
\newtheorem{corollary}{Corollary}
\newtheorem{proposition}{Proposition}
\theoremstyle{definition}
\newtheorem{example}{Example}
\theoremstyle{definition}
\newtheorem{exercise}{Exercise}
\theoremstyle{remark}
\newtheorem*{remark}{Remark}
\newtheorem*{solution}{Solution}
\begin{document}

\maketitle

\setcounter{secnumdepth}{0}



Previous research conducted on Judgments of Associative Memory (JAM) has
found that these judgments tend to be stable and highly generalizable
across varying contexts (Maki, 2007a, 2007b; Valentine \& Buchanan,
2013). The JAM task can be viewed as a manipulation of the traditional
Judgment of Learning task (JOL). In a JOL task, participants are
presented with cue-target word pairs and are asked to make a judgment
(typically, on a scale of zero to 100) of how accurately they would be
able to respond with the proper target word based on the presentation of
a particular cue word (Dunlosky \& Nelson, 1994; Nelson \& Dunlosky,
1991). JAM tasks expand upon this concept by changing the focus of the
judgments performed by participants. When presented with the item pair,
such as \emph{cheese-mouse}, participants are asked to judge the number
of people out of 100 who would respond with the pair's target word if
they were only shown the cue word (Maki, 2007a).

This process mimics the creation of associative words norms (i.e.,
forward strength; D. L. Nelson, McEvoy, and Schreiber (2004)). As such,
these judgments can be viewed as the participants' approximations of how
associatively related they perceive the paired items to be. The JAM
function can then be created by plotting participants' judgments against
the word's normed associative strength and calculating a line of best
fit. This fit line typically displays a high intercept (bias) and a
shallow slope (sensitivity), meaning that participants are biased
towards overestimating the associative relatedness between word pairs,
and show difficulties differentiating between different amounts of item
relatedness (Maki, 2007a). These results are often found in JOL research
(Koriat \& Bjork, 2005), and they are highly stable across contexts and
instructional manipulation (Valentine \& Buchanan, 2013).

Building upon this research, we initially explored recall accuracy
within the context of word pair judgments, while also expanding the JAM
task to incorporate judgments of semantic and thematic memory. In the
pilot study, 63 word-pairs of varying associative, semantic, and
thematic overlap were created and arranged into three blocks, consisting
of 21 word-pairs each. Associative overlap was measured with forward
strength (FSG; D. L. Nelson et al., 2004), semantic overlap was measured
with cosine (COS; Buchanan, Holmes, Teasley, \& Hutchison, 2013; McRae,
Cree, Seidenberg, \& McNorgan, 2005; Vinson \& Vigliocco, 2008), and
thematic relatedness between pairs was measured with latent semantic
analysis (LSA; Landauer \& Dumais, 1997; Landauer, Foltz, \& Laham,
1998). Participants then judged the word-pairs in three blocks based on
instructions explaining either an associative, semantic, or thematic
relationship between words. After completing the judgment phase,
participants then completed a cued recall task in which they were
presented with the cue word from each of the previously presented word
pairs and were asked to complete each pair with the missing target
({\textbf{???}}). Significant three-way interactions were found between
database norms when predicting judgments and recall. When semantic
overlap was low, thematic and associative strength were competitive,
with increases in thematic overlap decreasing the strength of
associative overlap as a predictor. However, this trend saw a reversal
when semantic overlap was high, with thematic and associative strength
complimenting one another. Overall, our findings from this study
indicated the degree to which the processing of associative, semantic,
and thematic information impacts retrieval and judgment making, while
also displaying the interactive relationship that exists between these
three types of information.

The proposed study seeks to expand upon this work by extending the
original analysis to include multiple single word norms. These norms
provide information about different \enquote{neighborhoods} of concept
information. Broadly speaking, they can be separated into one of three
categories. Base values refer to norms which capture information based
on a word's structure. These include part of speech, word frequency, and
the number of syllables, morphemes, and phonemes that comprise a word.
Rated values refer to age of acquisition, concreteness, imageability,
valence, and familiarity. Finally, we seek to examine norms that provide
information about the connections a word shares with others based on
context. These norms include orthographic neighborhood, phonographic
neighborhood, cue and target set sizes, and feature set size. These
values and their importance are explained below.

First, we are interested in assessing the impact of base word norms.
Chief amongst these is word frequency. Several sets of norms currently
exist for measuring the frequency with which words occur in everyday
language, and it is important to determine which of these offers the
best representation of everyday language. One of the most commonly used
collections of these norms is the Kucera and Francis (1967) frequency
norms. This set consists of frequency values for words, which were
generated by analyzing books, magazines, and newspapers. However, the
validity of using these norms has been questioned on factors such as the
properties of the sources analyzed, the size of the corpus analyzed, and
the overall age of these norms. First, these norms were created from an
analysis of written text. It is important to keep in mind that
stylistically, writing tends to be more formal than everyday language
and as a result, it may not be the best approximation of it (Brysbaert
\& New, 2009). Additionally, these norms were generated fifty years ago,
meaning that these norms may not accurately reflect the current state of
the English language. As such, the Kucera and Francis (1967) norms,
while popular, may not be the best choice for researchers interested in
gauging the effects of word frequency.

Several viable alternatives to the Kucera and Francis (1967) frequency
norms now exist. One popular method is to use frequency norms obtained
from the HAL corpus, which consists of 131 million words (Burgess \&
Lund, 1997; Lund \& Burgess, 1996). Other collections of frequency norms
include CELEX (Baayen, Piepenbrock, \& Gulikers, 1995) based on written
text, the Zeno frequency norms (Zeno, Ivens, Millard, \& Duvvuri, 1995)
created from American children's textbooks, and Google Book's collection
of word frequencies derived from 131 billion words taken from books
published in the United States (see Brysbaert, Keuleers, and New (2011)
for an overview and comparison of these norms). For the present study,
we plan to use data taken from the both the SUBTLEX project (Brysbaert
\& New, 2009), which is a collection of frequency norms derived from a
corpus of approximately 51 million words, which were generated from
movie and television subtitles and the HAL corpus. SUBTLEX norms are
thought to better approximate everyday language, as lines from movies
and television tend to be more reflective of everyday speech than
writing samples. Additionally, the larger corpus size of both SUBTLEX
and HAL contributes to the validity of these norms compared to Kucera
and Francis (1967) frequency norms.

Next, we are interested in testing the effects of several measures of
lexical information related to the physical make-up of words. These
measures include the numbers of phonemes, morphemes, and syllables that
comprise each word as well as its part of speech. The number of phonemes
refers to the number of individual sounds that comprise a word (i.e.,
the word \emph{cat} has three phonemes, each of which correspond to the
sounds its letters make), while the term morpheme refers to the number
of sound units that contain meaning. \emph{Drive} contains one morpheme,
while \emph{driver} contains two. Morphemes typically consist of root
words and their affixes. Additionally, word length (measured as the
number of individual characters a word consists of) and the number of
syllables a word contains will be investigated, as previous research has
suggested that the number of syllables may play a role in processing
time. In general, longer words require longer processing time (Kuperman,
Stadthagen-Gonzalez, \& Brysbaert, 2012), and shorter words tend to be
more easily remembered (Cowan, Baddeley, Elliott, \& Norris, 2003).
Finally, we are interested in the part of speech of each word, as nouns
are often easier to remember ({\textbf{???}}).

Third, we will examine the effects of norms measuring word properties
that are rated by participants. The first of these is age of
acquisition, which is a measure of the age at which a word is learned.
This norm is measured by presenting participants with a word and having
them enter the age (in years) in which they believe that they would have
learned the word (Kuperman et al., 2012). Age of acquisition ratings
have been found to be predictive of recall; for example, Dewhurst,
Hitch, and Barry (1998) found recall to be higher for late acquired
words. Also of interest are measures of a word's valence, which refers
to its intrinsic pleasantness or perceived positiveness
({\textbf{???}}). Valence ratings are important across multiple
psycholinguistic research settings. These include research on emotion,
the impact of emotion of lexical processing and memory, estimating the
sentiments of larger passages of text, and estimating the emotional
value of new words based on valence ratings of semantically similar
words (see Warriner, Kuperman, and Brysbaert (2013) for a review). The
next of these rated measures is concreteness, which refers to the degree
that a word relates to a perceptible object (Brysbaert, Warriner, \&
Kuperman, 2013). Similar to concreteness, imageability is described as
being a measure of a word's ability to generate a mental image
(Stadthagen-Gonzalez \& Davis, 2006). Both imageability and concreteness
have been linked to recall, as items rated higher in these areas tend to
be more easily recalled (Nelson \& Schreiber, 1992). Finally,
familiarity norms can be described as an application of word frequency.
These norms measure the frequency of exposure to a particular word
(Stadthagen-Gonzalez \& Davis, 2006).

The final group of norms that will be investigated are those which
provide information based on connections with neighboring words.
Phonographic neighborhood refers to refers to the number of words that
can be created by changing one sound in a word (i.e., \emph{cat} to
\emph{kite}). Similarly, orthographic neighborhood refers to the number
of words created by changing a single letter in word (i.e., \emph{cat}
to \emph{bat}, Adelman \& Brown, 2007; Peereman \& Content, 1997).
Previous findings have suggested that the frequency of a target word
relative to that of its orthographic neighbors has an effect on recall,
increasing the likelihood of recall for that word (Carreiras, Perea, \&
Grainger, 1997). Additionally, both of measures have been found to
effect processing speed for items (Adelman \& Brown, 2007; Buchanan et
al., 2013; Coltheart, Davelaar, Jonasson, \& Besner, 1977). Next, we are
interested in examining two single word norms that are directly related
to item associations. These norms measure the number of associates a
word shares connections with. Cue set size refers to the number of cue
words that a target word is connected to, while target set size is a
count of the number of target words a cue word is connected to
(Schreiber \& Nelson, 1998). Previous research has shown evidence for a
cue set size effect in which cue words that are linked to a larger
number of associates (target words) are less likely to be recalled than
cue words linked to fewer target words (D. L. Nelson, Schreiber, \& Xu,
1999). As such, feature list sizes will be calculated for each word
overlap norm from the Buchanan et al. (2013) semantic feature norm set.

In summary, this study seeks to expand upon previous work by examining
how single word norms belonging to these three neighborhoods of item
information impact the accuracy of item judgments and recall. These
findings will be assessed within the context of associative, semantic,
and thematic memory systems. Specifically, we utilize a three-tiered
view of the interconnections between these systems as it relates to
processing concept information. First, semantic information is
processed, which provides a means for categorizing concepts based on
feature similarity. Next, processing moves into the associative memory
network, where contextual information pertaining to the items is added.
Finally, the thematic network incorporates information from both the
associative and semantic networks to generate a mental representation of
the concept containing both the items meaning and its place in the
world.

Therefore, the present study has two aims. First, we seek to replicate
the interaction results from the previous study using a new set of
stimuli. Second, we wish to expand upon these findings by extending the
analysis to include neighborhood information for the item pairs. The
extended analysis will be analyzed by introducing the different types
single word norms through a series of steps based on the type of
neighborhood they belong to. First, base word norms will be analyzed.
Next, measures of word ratings will be analyzed. Third, single word
norms measuring connections between concepts will be analyzed. Finally,
network norms and their interactions will be reanalyzed. The end goal is
to determine both which neighborhood of norms have the greatest overall
impact on recall and judgment ability, and to further assess the impact
of network connections after controlling for the various neighborhoods
of single word information.

\section{Methods}\label{methods}

\subsection{Participants}\label{participants}

A power analysis was conducted using the \emph{simr} package in \emph{R}
(Green \& MacLeod, 2016), which uses simulations to calculate power for
mixed linear models created from the \emph{lme4} and \emph{nlme}
packages (D. Bates, Machler, Bolker, \& Walker, 2015; Pinheiro, Bates,
Debroy, Sarkar, \& R Core Team, 2017). The results of this analyses
suggested a minimum of 35 participants was required to find an effect at
80\% power. However, because power often is underestimated
({\textbf{???}}; Brysbaert \& Stevens, 2018), we plan to extend the
analysis to include approximately 200 participants, a number determined
by the amount of available funding. Participants will be recruited from
Amazon's Mechanical Turk, which is a website where individuals can host
projects and be connected with a large respondent pool who complete
tasks for small amounts of money (Buhrmester, Kwang, \& Gosling, 2011).
Participants will be paid \$2.00 for their participation. Participant
responses will be screened for a basic understanding of study
instructions and automated survey responses.

\subsection{Materials}\label{materials}

First, mimicking the design of the original pilot study, sixty-three
word pairs of varying associative, semantic, and thematic overlap were
created to use as stimuli. These word pairs were created using the
Buchanan et al. (2013) word norm database. Next, neighborhood
information for all cue and target items was collected. Word frequency
was collected from the SUBTLEX project (Brysbaert \& New, 2009) and the
HAL corpus (Burgess \& Lund, 1997). Part of speech, word length, and the
number of morphemes, phonemes, and syllables of each item was derived
from the Buchanan et al. (2013) word norms (originally contained in The
English Lexicon Project, Balota et al., 2007). For items with multiple
parts of speech (for example, \emph{drink} can refer to both a beverage
and the act of drinking a beverage), the most commonly used form was
used. Following the design of Buchanan et al. (2013), this part of
speech was determined using Google's define feature. Concreteness, cue
set size, and target set size were taken from the South Florida Free
Association Norms (D. L. Nelson et al., 2004). Feature set size (i.e.,
the number of features listed as part of the definition of a concept)
and cosine set size (i.e., number of semantically related words above a
cosine of zero) were calculated from ({\textbf{???}}). Imageability and
familiarity norms were taken from the Toglia and colleagues set of
semantic word norms (Toglia, 2009; Toglia \& Battig, 1978). Age of
acquisition ratings were pulled from the Kuperman et al. (2012)
database. Finally, valence ratings for all items were obtained from the
Warriner et al. (2013) norms. Stimuli information for cue and target
words can be found in Tables @ref:(tab:stim-table-cue) and
@ref:(tab:stim-table-target).

After gathering neighborhood information, network norms measuring
associative, semantic, and thematic overlap were generated for each
pair. Forward strength (FSG) was used as a measure of associative
overlap. FSG is a value ranging from zero to one which measures of the
probability that a cue word will elicit a particular target word in
response to it (D. L. Nelson et al., 2004). Cosine (COS) strength was
used to measure semantic overlap between concepts (Buchanan et al.,
2013; McRae et al., 2005; Vinson \& Vigliocco, 2008). As with FSG, this
value ranges from zero to one, with higher values indicating more shared
features between concepts. Finally, thematic overlap was measured with
Latent Semantic Analysis (LSA), which is a measure generated based upon
the co-occurrences of words within a document (Landauer \& Dumais, 1997;
Landauer et al., 1998). Like the measures of associative and semantic
overlap, LSA values range from zero to one, with higher values
indicating higher co-occurrence between items. The selected stimuli
contained a range of values across both the network and neighborhood
norms. As with the previous study, stimuli will be arranged into three
blocks, with each block consisting of 21 word pairs. The blocks will be
structured to have seven words of low COS (0 - .33), medium COS (.34 -
.66), and high COS (.67 - 1). COS was chosen due to both limitations
with the size of the available dataset across all norm sets, and the
desire to recreate the selection process used for the previous study.
The result of this selection process is that values for the remaining
network norms (FSG and LSA) and information neighborhood norms will be
contingent upon the COS strengths of the selected stimuli. To counter
this, we selected stimuli at random based on the different COS groupings
so as to cover a broad range of FSG, LSA, and information neighborhood
values. Stimuli information for word pair norms can be found in Table
@ref:(tab:stim-table-network). All stimuli and their raw values can be
found at \url{https://osf.io/j7qtc/}.

\begin{table}[tbp]
\begin{center}
\begin{threeparttable}
\caption{\label{tab:stim-table-network}Summary Statistics for Network Norms}
\begin{tabular}{llcccc}
\toprule
Variable & \multicolumn{1}{c}{Citation} & \multicolumn{1}{c}{Mean} & \multicolumn{1}{c}{SD} & \multicolumn{1}{c}{Min} & \multicolumn{1}{c}{Max}\\
\midrule
FSG & Nelson, McEvoy, and Schrieber, 2004 & 0.13 & 0.19 & 0.01 & 0.83\\
COS & Maki, McKinley, and Thompson, 2004 & 0.42 & 0.29 & 0.00 & 0.84\\
LSA & Landauer and Dumais, 1997 & 0.38 & 0.20 & 0.05 & 0.88\\
\bottomrule
\addlinespace
\end{tabular}
\begin{tablenotes}[para]
\textit{Note.} COS: Cosine, FSG: Forward Strength, LSA: Latent Semantic Analysis.
\end{tablenotes}
\end{threeparttable}
\end{center}
\end{table}

\begin{table}[tbp]
\begin{center}
\begin{threeparttable}
\caption{\label{tab:stim-table-cue}Summary Statistics of Single Word Norms for Cue Items}
\begin{tabular}{llcccc}
\toprule
Variable & \multicolumn{1}{c}{Citation} & \multicolumn{1}{c}{Mean} & \multicolumn{1}{c}{SD} & \multicolumn{1}{c}{Min} & \multicolumn{1}{c}{Max}\\
\midrule
QSS & Nelson et al., 2004 & 14.76 & 4.45 & 4.00 & 24.00\\
Concreteness & Nelson et al., 2004 & 5.35 & 1.00 & 1.98 & 7.00\\
HAL Frequency & Lund and Burgess, 1996 & 9.34 & 1.67 & 6.26 & 13.39\\
SUBTLEX Frequency & Brysbaert and New, 2009 & 3.15 & 0.74 & 1.76 & 5.20\\
Length & Buchanan et al., 2013 & 4.90 & 1.50 & 3.00 & 10.00\\
Ortho N & Buchanan et al., 2013 & 7.44 & 5.91 & 0.00 & 19.00\\
Phono N & Buchanan et al., 2013 & 19.00 & 15.11 & 0.00 & 51.00\\
Phonemes & Buchanan et al., 2013 & 3.94 & 1.39 & 2.00 & 9.00\\
Syllables & Buchanan et al., 2013 & 1.35 & 0.60 & 1.00 & 3.00\\
Morphemes & Buchanan et al., 2013 & 1.10 & 0.30 & 1.00 & 2.00\\
AOA & Kuperman et al., 2012 & 5.15 & 1.53 & 2.47 & 8.50\\
Valence & Warriner et al., 2013 & 5.77 & 1.23 & 1.91 & 7.72\\
Imageability & Toglia and Battig, 1978 & 5.52 & 0.68 & 3.22 & 6.61\\
Familiarity & Toglia and Battig, 1978 & 6.17 & 0.28 & 5.58 & 6.75\\
FSS & Buchanan et al., 2013 & 17.37 & 11.61 & 5.00 & 48.00\\
COSC & Buchanan et al., 2013 & 87.25 & 71.33 & 3.00 & 347.00\\
\bottomrule
\addlinespace
\end{tabular}
\begin{tablenotes}[para]
\textit{Note.} QSS: Cue Set Size, TSS: Target Set Size, Ortho N: Orthographic Neighborhood Size, Phono N: Phonographic Neighborhood Size, AOA: Age of Acquisition, FSS: Feature Set Size, COSC: Cosine Connectedness
\end{tablenotes}
\end{threeparttable}
\end{center}
\end{table}

\begin{table}[tbp]
\begin{center}
\begin{threeparttable}
\caption{\label{tab:stim-table-target}Summary Statistics of Single Word Norms for Target Items}
\begin{tabular}{llcccc}
\toprule
Variable & \multicolumn{1}{c}{Citation} & \multicolumn{1}{c}{Mean} & \multicolumn{1}{c}{SD} & \multicolumn{1}{c}{Min} & \multicolumn{1}{c}{Max}\\
\midrule
TSS & Nelson et al., 2004 & 15.44 & 4.86 & 5.00 & 26.00\\
Concreteness & Nelson et al., 2004 & 5.40 & 1.01 & 1.28 & 7.00\\
HAL Frequency & Lund and Burgess, 1996 & 9.78 & 1.52 & 6.05 & 13.03\\
SUBTLEX Frequency & Brysbaert and New, 2009 & 3.34 & 0.64 & 1.59 & 4.74\\
Length & Buchanan et al., 2013 & 4.62 & 1.67 & 3.00 & 10.00\\
Ortho N & Buchanan et al., 2013 & 9.02 & 7.77 & 0.00 & 29.00\\
Phono N & Buchanan et al., 2013 & 21.51 & 16.71 & 0.00 & 59.00\\
Phonemes & Buchanan et al., 2013 & 3.70 & 1.50 & 1.00 & 10.00\\
Syllables & Buchanan et al., 2013 & 1.25 & 0.54 & 1.00 & 3.00\\
Morphemes & Buchanan et al., 2013 & 1.05 & 0.21 & 1.00 & 2.00\\
AOA & Kuperman et al., 2012 & 4.87 & 1.56 & 2.50 & 9.16\\
Valence & Warriner et al., 2013 & 5.84 & 1.27 & 1.95 & 7.89\\
Imageability & Toglia and Battig, 1978 & 5.50 & 0.71 & 2.95 & 6.43\\
Familiarity & Toglia and Battig, 1978 & 6.28 & 0.32 & 5.19 & 6.85\\
FSS & Buchanan et al., 2013 & 16.70 & 11.62 & 5.00 & 54.00\\
COSC & Buchanan et al., 2013 & 91.71 & 79.52 & 3.00 & 322.00\\
\bottomrule
\addlinespace
\end{tabular}
\begin{tablenotes}[para]
\textit{Note.} QSS: Cue Set Size, TSS: Target Set Size, Ortho N: Orthographic Neighborhood Size, Phono N: Phonographic Neighborhood Size, AOA: Age of Acquisition, FSS: Feature Set Size, COSC: Cosine Connectedness
\end{tablenotes}
\end{threeparttable}
\end{center}
\end{table}

\subsection{Procedure}\label{procedure}

This study will be divided into three sections. First, participants will
be presented with word pairs and will be asked to judge how related the
items are to one another. This section will comprise three blocks, with
each block containing 21 word pairs. Each item block will be preceded by
a set of instructions explaining one of the three types of
relationships. Participants will also be provided with examples
illustrating the type of relationship to be judged. The associative
instructions explain associative relationships between concepts, how
these relationships can be strong or weak, and the role of free
association tasks in determining the magnitude of these relationships.
The semantic instructions will provide participants with a brief
overview of how words can be related by meaning and will give
participants examples of item pairs with low and high levels of semantic
overlap. Finally, the thematic instructions will explain how concepts
can be connected by overarching themes. These instruction sets are
modeled after Buchanan (2010) and Valentine and Buchanan (2013).

Participants will then rate the relatedness of the word pairs based on
the set of instructions they receive at the start of each judgment
block. These judgments will be made using a scale of zero (no
relatedness between pairs) to one hundred (a perfect relationships).
Judgments were recorded by the participant typing it into the survey.
Participants will complete each of the three judgment blocks in this
manner, with judgment instructions changing with each block. Three
versions of the study will be created to counterbalance the order in
which judgment blocks appear. Stimuli are counterbalanced across blocks,
such that each word pair is seen once per subject but evenly spread
across all three judgment types. Word pairs are randomized within each
block. Participants will be randomly assigned to survey conditions.
After completing the judgment blocks, participants will be presented
with a short distractor task to account for recency effects. This
section will be timed to last two minutes and will task participants
with alphabetizing a scrambled list of the fifty U.S. states. Once two
minutes elapses, participants will automatically progress to a cued
recall task, in which they will be presented with each of the 63 cues
that had previously been judged as cue-target pairs. Participants will
be asked to complete each word pair with the appropriate target word,
based on the available cue word. Presentation of these pairs will be
randomized, and participants will be informed that there is no penalty
for guessing. The Qualtrics surveys are uploaded at
\url{https://osf.io/j7qtc/}.

\section{Results}\label{results}

First, the results from the recall section will be coded as zero for
incorrect responses and one for correct responses. NA will be used to
denote missing responses from participants who did not complete the
recall section. Responses that are words instead of numbers in the
judgment phase will be deleted and treated as missing data. Data will
then be screened for out of range judgment responses (i.e., responses
greater than 100). Recall and judgment scores will be screened for
outliers using Mahalanobis distance at \emph{p} \textless{} .001
({\textbf{???}}), and multicollinearity between predictor variables will
be measured with Pearson correlations. Data will then be screened for
assumptions of normality, linearity, homogeneity, and homoscedasticity.
Descriptive statistics of mean judgment and recall scores will be
reported for each judgment condition.

Multilevel modeling will then be used to analyze the data (Gelman, 2006)
to control for the nested structure of the data using the \emph{nlme}
library. Each participant's judgment and recall ratings will be treated
as a data point, using participants as a nested random intercept factor.
As part of our replication, we will reanalyze these new stimuli using
COS, FSG, LSA, and their interaction to predict judgments and recall
separately as the dependent variables. Just as in ({\textbf{???}}),
judgment condition was used as a control variable. Variables will be
mean centered prior to analysis to control for multicollinearity. If a
significant three-way interaction occurs, simple slopes analyses will be
used to explore that interaction. We will examine low (-1SD), average
(mean), and high (+1SD) COS values for two-way interactions of FSG and
LSA. If these values are significant, LSA will be further broken into
low, average, and high simple slopes to examine FSG. \(\alpha\) is set
to .05 for analyses. We predict that the interaction found previously
will replicate on a new set of stimuli.

A second set of analyses will be performed using the ({\textbf{???}})
stimuli set and this new stimuli set combined, examining the hypothesis
of interactive networks after controlling for base word norm
information. Stimuli sets from both studies will be combined to create a
larger range of stimuli and values across normed information. These
neighborhood norms will be added introduced into each model in steps,
after controlling for judgment condition. Initially, base word norms
will be added, followed by lexical information, rated properties, and
norms measuring neighborhood connections, as described in the
introduction and methods. Each set of variables will be used to predict
the dependent variables of judgment and recall, again as a multilevel
model. Each variable will be discussed in the step of the analysis it
was entered. We expect that many of these variables will significantly
predict judgments and recall, but do not predict which ones in
particular. Last, the interaction of network norms will be added to the
model with the prediction that the interaction of COS, FSG, and LSA may
be significant, even after controlling for single concept information.

This analysis plan was pre-registered as part of the Pre-Registration
Challenge through the Open Science Foundation and may be found at:
\url{https://osf.io/j7qtc/}.

\newpage

\section{References}\label{references}

\setlength{\parindent}{-0.5in} \setlength{\leftskip}{0.5in}

\hypertarget{refs}{}
\hypertarget{ref-Adelman2007}{}
Adelman, J. S., \& Brown, G. D. A. (2007). Phonographic neighbors, not
orthographic neighbors, determine word naming latencies.
\emph{Psychonomic Bulletin \& Review}, \emph{14}, 455--459.

\hypertarget{ref-Baayen1995}{}
Baayen, R. H., Piepenbrock, R., \& Gulikers, L. (1995). The CELEX
lexical database (CD-ROM). Philidelphia.

\hypertarget{ref-Balota2007}{}
Balota, D. A., Yap, M. J., Cortese, M. J., Hutchison, K. A., Kessler,
B., Loftis, B., \ldots{} Treiman, R. (2007). The English Lexicon
Project. \emph{Behavior Research Methods}, \emph{39}(3), 445--459.
doi:\href{https://doi.org/10.3758/BF03193014}{10.3758/BF03193014}

\hypertarget{ref-Bates2015}{}
Bates, D., Machler, M., Bolker, B., \& Walker, S. (2015). Fitting Linear
Mixed-Effects Models Using lme4. \emph{Journal of Statistical Software},
\emph{67}(1), 1--48.

\hypertarget{ref-Brysbaert2009}{}
Brysbaert, M., \& New, B. (2009). Moving beyond Kučera and Francis: A
critical evaluation of current word frequency norms and the introduction
of a new and improved word frequency measure for American English.
\emph{Behavior Research Methods}, \emph{41}(4), 977--990.
doi:\href{https://doi.org/10.3758/BRM.41.4.977}{10.3758/BRM.41.4.977}

\hypertarget{ref-Brysbaert2018}{}
Brysbaert, M., \& Stevens, M. (2018). Power Analysis and Effect Size in
Mixed Effects Models: A Tutorial. \emph{Journal of Cognition},
\emph{1}(1), 1--20.
doi:\href{https://doi.org/10.5334/joc.10}{10.5334/joc.10}

\hypertarget{ref-Brysbaert2011}{}
Brysbaert, M., Keuleers, E., \& New, B. (2011). Assessing the usefulness
of Google Books' word frequencies for psycholinguistic research on word
processing. \emph{Frontiers in Psychology}, \emph{2}(MAR), 1--8.
doi:\href{https://doi.org/10.3389/fpsyg.2011.00027}{10.3389/fpsyg.2011.00027}

\hypertarget{ref-Brysbaert2013}{}
Brysbaert, M., Warriner, A. B., \& Kuperman, V. (2013). Concreteness
ratings for 40 thousand generally known English word lemmas.
\emph{Behavior Research Methods}, \emph{41}, 977--990.

\hypertarget{ref-Buchanan2010}{}
Buchanan, E. M. (2010). Access into Memory: Differences in Judgments and
Priming for Semantic and Associative Memory. \emph{Journal of Scientific
Psychology.}, (March), 1--8. Retrieved from
\href{http://www.psyencelab.com/images/Access\%7B/_\%7Dinto\%7B/_\%7DMemory\%7B/_\%7D\%7B/_\%7DDifferences\%7B/_\%7Din\%7B/_\%7DJudgments\%7B/_\%7Dand\%7B/_\%7DPriming\%7B/_\%7Dfor\%7B/_\%7DSemantic\%7B/_\%7Dand\%7B/_\%7DAssociative\%7B/_\%7DMemory.pdf}{http://www.psyencelab.com/images/Access\{\textbackslash{}\_\}into\{\textbackslash{}\_\}Memory\{\textbackslash{}\_\}\{\textbackslash{}\_\}Differences\{\textbackslash{}\_\}in\{\textbackslash{}\_\}Judgments\{\textbackslash{}\_\}and\{\textbackslash{}\_\}Priming\{\textbackslash{}\_\}for\{\textbackslash{}\_\}Semantic\{\textbackslash{}\_\}and\{\textbackslash{}\_\}Associative\{\textbackslash{}\_\}Memory.pdf}

\hypertarget{ref-Buchanan2013}{}
Buchanan, E. M., Holmes, J. L., Teasley, M. L., \& Hutchison, K. A.
(2013). English semantic word-pair norms and a searchable Web portal for
experimental stimulus creation. \emph{Behavior Research Methods},
\emph{45}(3), 746--757.
doi:\href{https://doi.org/10.3758/s13428-012-0284-z}{10.3758/s13428-012-0284-z}

\hypertarget{ref-Buhrmester2011}{}
Buhrmester, M., Kwang, T., \& Gosling, S. D. (2011). Amazon's Mechanical
Turk. \emph{Perspectives on Psychological Science}, \emph{6}(1), 3--5.
doi:\href{https://doi.org/10.1177/1745691610393980}{10.1177/1745691610393980}

\hypertarget{ref-Burgess1997}{}
Burgess, C., \& Lund, K. (1997). Representing abstract words and
emotional connotation in a high-dimensional memory space.
\emph{Proceedings of the Cognitive Science Society}, 61--66. Retrieved
from
\href{http://books.google.com/books?hl=en\%7B/\&\%7Dlr=\%7B/\&\%7Did=sQyJiDk45HEC\%7B/\&\%7Doi=fnd\%7B/\&\%7Dpg=PA61\%7B/\&\%7Ddq=representing+abstract+words+and+emotional+connotation+in+a+high+dimensional+memory+space\%7B/\&\%7Dots=59Iul8u8kD\%7B/\&\%7Dsig=-iCt-Jbg6O8i27OKajqGo\%7B/_\%7DADoko\%7B/\%\%7D5Cnpapers3://publication/uuid/FE5168D9-C7C7-4C0F}{http://books.google.com/books?hl=en\{\textbackslash{}\&\}lr=\{\textbackslash{}\&\}id=sQyJiDk45HEC\{\textbackslash{}\&\}oi=fnd\{\textbackslash{}\&\}pg=PA61\{\textbackslash{}\&\}dq=representing+abstract+words+and+emotional+connotation+in+a+high+dimensional+memory+space\{\textbackslash{}\&\}ots=59Iul8u8kD\{\textbackslash{}\&\}sig=-iCt-Jbg6O8i27OKajqGo\{\textbackslash{}\_\}ADoko\{\textbackslash{}\%\}5Cnpapers3://publication/uuid/FE5168D9-C7C7-4C0F}

\hypertarget{ref-Carreiras1997}{}
Carreiras, M., Perea, M., \& Grainger, J. (1997). Effects of
orthographic neighborhood in visual word recognition: cross-task
comparisons. \emph{Journal of Experimental Psychology. Learning, Memory,
and Cognition}, \emph{23}(4), 857--871.
doi:\href{https://doi.org/10.1037/0278-7393.23.4.857}{10.1037/0278-7393.23.4.857}

\hypertarget{ref-Coltheart1977}{}
Coltheart, M., Davelaar, E., Jonasson, T., \& Besner, D. (1977). Access
to the internal lexicon. In S. Dornic (Ed.), \emph{Attention and
performance vi} (pp. 535--555). Hillsdale, NJ: Earlbaum.

\hypertarget{ref-Cowan2003}{}
Cowan, N., Baddeley, A. D., Elliott, E. M., \& Norris, J. (2003). List
composition and the word length effect in immediate recall: A comparison
of localist and globalist assumptions. \emph{Psychonomic Bulletin and
Review}, \emph{10}(1), 74--79.
doi:\href{https://doi.org/10.3758/BF03196469}{10.3758/BF03196469}

\hypertarget{ref-Dewhurst1998}{}
Dewhurst, S. a., Hitch, G. J., \& Barry, C. (1998). Separate effects of
word frequency and age of acquisition in recognition and recall.
\emph{Journal of Experimental Psychology: Learning, Memory, and
Cognition}, \emph{24}(2), 284--298.
doi:\href{https://doi.org/10.1037/0278-7393.24.2.284}{10.1037/0278-7393.24.2.284}

\hypertarget{ref-Dunlosky1994a}{}
Dunlosky, J., \& Nelson, T. O. (1994). Does the sensitivity of judgments
of learning (JOLs) to the effects of various study activities depend on
when the JOLs occur?
doi:\href{https://doi.org/10.1006/jmla.1994.1026}{10.1006/jmla.1994.1026}

\hypertarget{ref-Gelman2006}{}
Gelman, A. (2006). Multilevel (Hierarchical) Modeling: What It Can and
Cannot Do. \emph{Technometrics}, \emph{48}(3), 432--435.
doi:\href{https://doi.org/10.1198/004017005000000661}{10.1198/004017005000000661}

\hypertarget{ref-Green2016}{}
Green, P., \& MacLeod, C. J. (2016). SIMR: An R Package for Power
Analysis of Generalized Linear Mixed Models by Simulation. \emph{Methods
in Ecology and Evolution}, \emph{7}(4), 493--498.

\hypertarget{ref-Koriat2005}{}
Koriat, A., \& Bjork, R. A. (2005). Illusions of competence in
monitoring one's knowledge during study. \emph{Journal of Experimental
Psychology: Learning, Memory, and Cognition}, \emph{31}(2), 187--194.
doi:\href{https://doi.org/10.1037/0278-7393.31.2.187}{10.1037/0278-7393.31.2.187}

\hypertarget{ref-Kucera1967}{}
Kucera, H., \& Francis, W. N. (1967). \emph{Computational analysis of
present-day English}. Providence, RI: Brown University Press.

\hypertarget{ref-Kuperman2012}{}
Kuperman, V., Stadthagen-Gonzalez, H., \& Brysbaert, M. (2012).
Age-of-acquisition ratings for 30,000 English words. \emph{Behavior
Research Methods}, \emph{44}(4), 978--990.
doi:\href{https://doi.org/10.3758/s13428-012-0210-4}{10.3758/s13428-012-0210-4}

\hypertarget{ref-Landauer1997}{}
Landauer, T. K., \& Dumais, S. T. (1997). A solution to Plato's problem:
The latent semantic analysis theory of acquisition, induction, and
representation of knowledge. \emph{Psychological Review}, \emph{104}(2),
211--240.
doi:\href{https://doi.org/10.1037//0033-295X.104.2.211}{10.1037//0033-295X.104.2.211}

\hypertarget{ref-Landauer1998}{}
Landauer, T. K., Foltz, P. W., \& Laham, D. (1998). An introduction to
latent semantic analysis. \emph{Discourse Processes}, \emph{25}(2),
259--284.
doi:\href{https://doi.org/10.1080/01638539809545028}{10.1080/01638539809545028}

\hypertarget{ref-Lund1996}{}
Lund, K., \& Burgess, C. (1996). Producing high-dimensional semantic
spaces from lexical co-occurrence. \emph{Behavior Research Methods,
Instruments, \& Computers}, \emph{28}(2), 203--208.
doi:\href{https://doi.org/10.3758/BF03204766}{10.3758/BF03204766}

\hypertarget{ref-Maki2007a}{}
Maki, W. S. (2007a). Judgments of associative memory. \emph{Cognitive
Psychology}, \emph{54}(4), 319--353.
doi:\href{https://doi.org/10.1016/j.cogpsych.2006.08.002}{10.1016/j.cogpsych.2006.08.002}

\hypertarget{ref-Maki2007}{}
Maki, W. S. (2007b). Separating bias and sensitivity in judgments of
associative memory. \emph{Journal of Experimental Psychology. Learning,
Memory, and Cognition}, \emph{33}(1), 231--7.
doi:\href{https://doi.org/10.1037/0278-7393.33.1.231}{10.1037/0278-7393.33.1.231}

\hypertarget{ref-McRae2005}{}
McRae, K., Cree, G. S., Seidenberg, M. S., \& McNorgan, C. (2005).
Semantic feature production norms for a large set of living and
nonliving things. \emph{Behavior Research Methods}, \emph{37}(4),
547--559.
doi:\href{https://doi.org/10.3758/BRM.40.1.183}{10.3758/BRM.40.1.183}

\hypertarget{ref-Nelson2004}{}
Nelson, D. L., McEvoy, C. L., \& Schreiber, T. A. (2004). The University
of South Florida free association, rhyme, and word fragment norms.
\emph{Behavior Research Methods, Instruments, \& Computers},
\emph{36}(3), 402--407.
doi:\href{https://doi.org/10.3758/BF03195588}{10.3758/BF03195588}

\hypertarget{ref-Nelson1999}{}
Nelson, D. L., Schreiber, T. A., \& Xu, J. (1999). Cue set size effects:
sampling activated associates or cross-target interference? \emph{Memory
\& Cognition}, \emph{27}(3), 465--477.
doi:\href{https://doi.org/10.3758/BF03211541}{10.3758/BF03211541}

\hypertarget{ref-Nelson1991}{}
Nelson, T. O., \& Dunlosky, J. (1991). When people's judgments of
learning (JOLs) are extremely acurate at predicting subsequent recall:
The delayed-JOL effect. \emph{Psychological Science}, \emph{2}(4),
267--270.
doi:\href{https://doi.org/10.1111/j.1467-9280.1991.tb00147.x}{10.1111/j.1467-9280.1991.tb00147.x}

\hypertarget{ref-Nelson1992}{}
Nelson, T. O., \& Schreiber, T. A. (1992). Word concreteness and word
structure as independent determinants of recall. \emph{Journal of Memory
and Language}, \emph{31}, 237--260.

\hypertarget{ref-Peereman1997}{}
Peereman, R., \& Content, A. (1997). Orthographic and phonological
neighborhoods in naming: Not all neighbors are equally influential in
orthographic space. \emph{Journal of Memory and Language}, \emph{37},
382--410.

\hypertarget{ref-Pinheiro2017}{}
Pinheiro, J., Bates, D., Debroy, S., Sarkar, D., \& R Core Team. (2017).
nlme: Linear and Nonlinear Mixed Effects Models. Retrieved from
\url{https://cran.r-project.org/package=nlme}

\hypertarget{ref-Schreiber1998}{}
Schreiber, T. A., \& Nelson, D. L. (1998). The relation between feelings
of knowing and the number of neighboring concepts linked to the test
cue. \emph{Memory \& Cognition}, \emph{26}(5), 869--83.
doi:\href{https://doi.org/10.3758/BF03201170}{10.3758/BF03201170}

\hypertarget{ref-Stadthagen-Gonzalez2006}{}
Stadthagen-Gonzalez, H., \& Davis, C. J. (2006). The Bristol norms for
age of acquisition, imageability, and familiarity. \emph{Behavior
Research Methods}, \emph{38}, 598--605.

\hypertarget{ref-Toglia2009}{}
Toglia, M. P. (2009). Withstanding the test of time: The 1978 semantic
word norms. \emph{Behavior Research Methods}, \emph{41}(2), 531--533.
doi:\href{https://doi.org/10.3758/BRM.41.2.531}{10.3758/BRM.41.2.531}

\hypertarget{ref-Toglia1978}{}
Toglia, M. P., \& Battig, W. F. (1978). \emph{Handbook of semantic word
norms}. Hillside, NJ: Earlbaum.

\hypertarget{ref-Valentine2013}{}
Valentine, K. D., \& Buchanan, E. M. (2013). JAM-boree: An application
of observation oriented modelling to judgements of associative memory.
\emph{Journal of Cognitive Psychology}, \emph{25}(4), 400--422.
doi:\href{https://doi.org/10.1080/20445911.2013.775120}{10.1080/20445911.2013.775120}

\hypertarget{ref-Vinson2008}{}
Vinson, D. P., \& Vigliocco, G. (2008). Semantic feature production
norms for a large set of objects and events. \emph{Behavior Research
Methods}, \emph{40}(1), 183--190.
doi:\href{https://doi.org/10.3758/BRM.40.1.183}{10.3758/BRM.40.1.183}

\hypertarget{ref-Warriner2013}{}
Warriner, A. B., Kuperman, V., \& Brysbaert, M. (2013). Norms of
Valence, Arousal, and Dominance for 13,915 English Lemmas.
\emph{Behavior Research Methods}, \emph{45}(4), 1191--1207.

\hypertarget{ref-Zeno1995}{}
Zeno, S. M., Ivens, S. H., Millard, R. T., \& Duvvuri, R. (1995).
\emph{The educators's word frequency guide}. Brewster, NY: Touchstone
Applied Science.






\end{document}
